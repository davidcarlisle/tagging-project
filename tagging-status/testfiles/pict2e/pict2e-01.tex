\DocumentMetadata
  {
    lang=en-US,
    pdfversion=2.0,
    pdfstandard=ua-2,
    tagging=on
  }
\documentclass{article}

\usepackage{pict2e,xcolor}

\begin{document}

\setlength\unitlength{.007\linewidth}
\begin{picture}[alt={my alt text}](100,100)
 \put(0,0){\framebox(100,100){}}
 \linethickness{2.6\unitlength}
 \put(0,0){\line(1,0){50}}
 \put(0,0){\line(2,1){50}}
 \put(0,0){\line(1,1){50}}
 \put(0,0){\line(1,2){50}}
 \put(0,0){\line(0,1){50}}
 \color{blue}
 \thicklines
 \put(50,0){\line(4,2){50}}
 \put(50,0){\line(2,4){50}}
 \put(50,0){\line(5,5){50}}
\end{picture}

\bigskip

\setlength{\unitlength}{1cm}
\begin{picture}[alt={second picture}](8,4)
  \thinlines % Start with thin lines
  \put(0,0){\vector(1,0){8}}  % x axis
  \put(0,0){\vector(0,1){4}}  % y axis
  \put(2,0){\line(0,1){3}}    % left side
  \put(4,0){\line(0,1){3.5}}  % right side
  \thicklines % Use thicker lines for the \qbezier commands
  \qbezier(2,3)(2.5,2.9)(3,3.25)
  \qbezier(3,3.25)(3.5,3.6)(4,3.5)
  \thinlines % Back to using thin lines
  \put(2,3){\line(4,1){2}}
  \put(4.5,2.5){\framebox{Trapezoidal Rule}}
\end{picture}

\bigskip

\setlength\unitlength{1cm}
\begin{picture}[alt={third picture}](5,4)
{\color{orange}\polygon*(0,0)(4,3)(5,0)}
{\linethickness{1pt}\polygon(0,0)(4,3)(5,0)}
\end{picture}

\end{document}
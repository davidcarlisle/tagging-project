\DocumentMetadata
  {
    lang=en-US,
    pdfversion=2.0,
    pdfstandard=ua-2,
    tagging=on
  }
\documentclass{article}

\usepackage{mathtools}
\UseName{sys_if_engine_opentype:T}{\usepackage{unicode-math}}

\title{mathtools tagging test - Controlling tags}

\newtagform{brackets}{[}{]}
\newtagform{brackets2}[\textbf]{[}{]}

\begin{document}

\begin{quote}\normalfont\itshape
\begin{equation*}
a=b \label{eq:example}\tag*{Q\&A}
\end{equation*}
See \ref{eq:example} or is it better with \refeq{eq:example}?
\end{quote}

\usetagform{brackets2}
\begin{equation}
E \neq m c^3
\end{equation}

\mathtoolsset{showonlyrefs,showmanualtags}
\usetagform{brackets}
\begin{gather}
a=a \label{eq:a} \\
b=b \label{eq:b} \tag{**}
\end{gather}
This should refer to the equation containing $a=a$: \eqref{eq:a}.
Then a switch of tag forms.
\usetagform{default}
\begin{align}
c&=c \label{eq:c} \\
d&=d \label{eq:d}
\end{align}
This should refer to the equation containing $d=d$: \eqref{eq:d}.
\begin{equation}
e=e
\end{equation}
Back to normal.\mathtoolsset{showonlyrefs=false}
\begin{equation}
f=f
\end{equation}

\end{document}
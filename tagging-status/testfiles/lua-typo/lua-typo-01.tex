% adapted from lua-typo-demo.tex
\DocumentMetadata
  {
    lang=en-US,
    pdfversion=2.0,
    pdfstandard=ua-2,
    tagging=on
  }
\documentclass[french,twoside]{article}
\usepackage{geometry}
\geometry{a6paper}
\usepackage{fourier-otf}
\usepackage{microtype}

\usepackage{babel}
\frenchsetup{og=«, fg=»}

\usepackage[All]{lua-typo}   % APRÈS babel / polyglossia

% Mots courts (deux lettres max) à éviter en fin de ligne
\luatypoOneChar{french}{'À à Ô'}
\luatypoTwoChars{french}{"Je Tu Il On"}

% Pour changer les couleurs :
%\usepackage[svgnames]{xcolor}
%\LuaTypoSetColor1{Fuchsia}
%\LuaTypoSetColor2{ForestGreen}
%\LuaTypoSetColor5{magenta}
%\LuaTypoSetColor6{yellow}

% Réglage ligne finale trop courte
\setlength{\luatypoLLminWD}{3em}

%\luatypoPageMin=5
%\luatypoHyphMax=2

\parindent=1em

\title{lua-typo tagging test}

\begin{document}

Le plus effroyablement démuni des pauvres peut toujours espérer
\footnote{Overfull hbox  à la ligne précédente.}
décrocher un jour le gros lot de la tombola organisée par
l’association des Pauvres Effroyablement Démunis.
%
Le laideron lui, n’a d’autre échappatoire que de
ronger son frein d’un bec-de-lièvre machinal, de baisser ses yeux quelconques
aux abords des mirroirs qui l’insultent ou de se foutre à l’eau au risque
d’effaroucher les murènes.

Quelquefois je trouve que Dieu pousse un peu.

« Les hommes naissent libres et égaux en droit. »

Qu’on me pardonne mais c’est un phrase que j’ai beaucoup de mal à dire sans
rire. % : « Les hommes naissent libres et égaux en droit. »

Prenons une star, une belle star. Elle est belle.

La beauté. Existe-t-il au monde un privilège plus exorbitant que la beauté ?

Par sa beauté, cette femme n’est-elle pas un peu plus libre
et un peu plus égale , dans le grand combat pour survivre,
que l’\textit{Homo sapiens} moyen qui passe sa vie à se courir
après la queue en attendant la mort ?

Quel profond imbécile aurait l’outrecuidance de soutenir, au nom des grands
principes révolutionnaires, que l’immonde boudin trapu qui m’a
%collé une contredanse
lâchement verbalisé
tout-à-l’heure
possède les mêmes armes
%pour assoir son bonheur terrestre
que la grande fille féline aux charmes troubles où l’œil se
pose et chancelle avec une lubricité contenue !
%(Difficilement contenue.)  % Variantes pour mettre 1 ou 2 car. en bout de ligne
%(Difficilement contenue hein). Je dois le dire.
%(Difficilement contenue hein). On peut le dire.
%(Difficilement contenue hein). Il faut l’admettre.
(Difficilement contenue hein). À voir.

Quand on a vos yeux, madame, quand on a votre bouche, votre grain de peau, la
légèreté diaphane de votre démarche et la longueur émouvante de vos cuisses,
c’est une banalité de dire qu’on peut facilement traverser la vie %l’existence
à l’abri des cabats trop lourds gorgés de poireaux, à l’écart de l’uniforme de
contractuelle.

\end{document}
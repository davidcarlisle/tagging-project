% this is tugboat's article template
\DocumentMetadata
  {
    lang=en-US,
    pdfversion=2.0,
    pdfstandard=ua-2,
    tagging=on
  }
\documentclass{ltugboat}
%\usepackage[T1]{fontenc}
\usepackage{graphicx}
\usepackage{microtype}
\usepackage{hyperref}

\title{Example \TUB\ article}

% repeat info for each author; comment out items that don't apply.
\author{First Last}
\address{Street Address \\ Town, Postal \\ Country}
\netaddress{user (at) example dot org}
\personalURL{https://example.org/~user/}
%\ORCID{0}

% Please state if you'd like to receive a physical copy of the TUGboat
% issue or if electronic access suffices.  If you want a physical issue,
% please include the mailing address we should use, as a comment if
% you prefer it not be printed.

\begin{document}
\maketitle

\begin{abstract}
Please write an abstract. Please use only standard \LaTeX\ and \TUB\
macros in the abstract, not article-specific abbreviations or other
macros; that helps us make the web pages.
\end{abstract}

\section{Introduction}

This is an example article for \TUB, linked from
\url{https://tug.org/TUGboat/location.html}.

\section{Basic packages and hyperlinks}

The standard \texttt{graphicx} and \texttt{xcolor} packages work fine
for image and color handling, respectively, as does the
\texttt{hyperref} package for active urls. \TUB\ is produced using
\acro{PDF} files exclusively, but conversion from \DVI\ is fine and supported.

Please use the standard \cs{url} command for urls:
\begin{verbatim}
\url{https://some/url} % yes
\end{verbatim}
For \TUB, we prefer not to print the `\texttt{https://}', but to include
the protocol in other cases, so we usually do \verb|\def\url{\tburl}|,
which takes care of that, while keeping the source using the standard
\cs{url} command.

In any case, please do \emph{not} hide hyperlinks via \cs{href}, as in:
\begin{verbatim}
\href{https://some/url}{some text} % no!
\end{verbatim}
When printed on paper, the url will not be visible and readers will
not know it exists.

\section{Abbreviation macros and typing}

The \texttt{ltugboat} class provides many abbreviation commands; here
are a few of the most common:

% verbatim blocks are often better in \small
\begin{verbatim}[\small]
\AllTeX \AMS \BibTeX \Cplusplus \CTAN \DVI
\HTML \LaTeXe \macOS \MathML \MF \PDF \PS
\TUB \TUG \tug \WEB \XeLaTeX \XeTeX \XML
\end{verbatim}

A few other typing conventions:

\begin{itemize}
\item For an em-dash with our spacing (preferred to \verb|---|):
\cs{Dash}, with output\Dash like this.

\item For initialisms in all caps:
\verb|\acro{FRED}|,\\ with output: \acro{FRED}.

\item A literal control sequence:
\verb|\cs{fred}|,\\ with output: \cs{fred}.

\item A syntactic metavariable:
\verb|\meta{fred}|,\\ with output: \meta{fred}.

\item A title:
\verb|\titleref{Book of Fred}|,\\ with output: \titleref{Book of Fred}.
\end{itemize}

We recommend using \cs{begin}\tubbraced{verbatim}\texttt{[\small] ...}
\cs{end}\tubbraced{verbatim} for code blocks, since we prefer not to
colorize code when printed. But if you want to have some font changes,
our recommended settings for the \texttt{listings} package are in the
\texttt{ltubguid} manual mentioned below.

Please put punctuation outside quotes, ``like this'', unless the
punctuation is actually part of the quoted material.

Also, please put punctuation after footnotes.

\section{Figures}

For \TUB, the standard \texttt{figure} environment produces a
column-width figure; this is desirable when at all possible. The
\texttt{figure*} environment produces a full-width (across both columns)
figure when needed. Analogously for \texttt{table} and \texttt{table*}.

Please put captions below figures, but above tables.

Don't worry overmuch about figure placements, as they will likely change
with editing.

\begin{figure}
This is a column-width figure, made with \\
\cs{begin}\tubbraced{figure}. Use \tubbraced{figure*} for a full-width
(double-column) figure.
%
\caption{Caption for column-width figure.}
\label{fig.example}
\end{figure}

%\begin{figure*}
%A full-width figure, made with \cs{begin}\tubbraced{figure*}.
%\caption{Caption for full-width figure.}
%\label{fig.fullwidth}
%\end{figure*}

\section{References}

For references to other issues of \TUB, please use the format
\textsl{volno}:\textsl{issno}, e.g., ``\TUB\ 32:1''.

The primary \TUB\ style documentation is the \texttt{ltubguid} manual,
available online at \tbsurl{ctan.org/pkg/tugboat} or locally with
\texttt{texdoc tugboat}. For general \CTAN\ package references, we
recommend that form, using \texttt{/pkg/}. If you need to refer to a
specific file on \CTAN, use:\\
\texttt{https://mirror.ctan.org/\textsl{path}}

We recommend using \BibTeX\ (but don't require it), with the
\texttt{tugboat} \BibTeX\ style. The \texttt{biblio} directory on \CTAN,
\tbsurl{ctan.org/pkg/biblio}, provides files \texttt{tugboat.bib} with a
complete bibliography of \TUB, \texttt{texbook3.bib} with many common
\TeX-related books and articles, and plenty more.

Email \verb|tugboat@tug.org| with any questions.

\bibliographystyle{tugboat} % tugboat's bibtex style
\nocite{book-minimal}       % make the example bibliography non-empty
\bibliography{xampl}        % xampl.bib comes with bibtex

\makesignature

\newpage

\AllTeX\par
\AMS\par
\AmSTeX\par
\aw\par
\API\par
\AW\par
\BibTeX\par
\CandT\par
\ConTeXt\par
\Cplusplus\par
\DTD\par
\DVD\par
\DVI\par
\DVIPDFMx\par
\DVItoVDU\par
\ECMA\par
\EPS\par
\eTeX\par
\ExTeX\par
\Ghostscript\par
\Hawaii\par
\HTML\par
\ISBN\par
\ISO\par
\ISSN\par
\JTeX\par
\JoT\par
\LaTeX\par
\LyX\par
\macOS\par
\MacOSX\par
\MathML\par
\Mc\par
\MF\par
\mf\par
\MFB\par
\MP\par
\mp\par
\OMEGA\par
\OCP\par
\OOXML\par
\OTP\par
\mtex\par
\NTS\par
\pcMF\par
\PCTeX\par
\pcTeX\par
\Pas\par
\PiCTeX\par
\plain\par
\POBox\par
\PS\par
\SC\par
\SGML\par
\SliTeX\par
\slMF\par
\stTeX\par
\SVG\par
\TANGLE\par
\TB \par
\TeX\par
\TeXhax\par
\TeXMaG\par
\TeXtures\par
\TeXXeT\par
\Thanh\par
\TFM\par
\TUB\par
\TUG\par
\UNIX\par
\VAX\par
\VnTeX\par
\VorTeX\par
\XeT\par
\XeTeX\par
\XeLaTeX\par
\XML\par
\WEB\par
\WEAVE\par
\WYSIWYG\par

\bull\par
3\cents\par
3\Dag\par
\careof\par
\sfrac{1}{3}\par
\cs{cmd}\par
\meta{arg}\par
\dash\par
\Dash\par
\hyph\par
\slash\par
\nth{3}\par

\end{document}
% adapted from 2-footnote_spacing.tex
% errors
\DocumentMetadata
  {
    lang=en-US,
    pdfversion=2.0,
    pdfstandard=ua-2,
    tagging=on
  }
\documentclass{article}

\usepackage[colorlinks]{hyperref}
\usepackage[detect-all]{siunitx}

\usepackage[series={A,B,C},noend,noeledsec,noledgroup]{reledmac}

% THE APPARATI ARE PARAGRAPHED
\Xarrangement{paragraph}


%VERTICAL SPACES BEFORE APPARATUS
\Xprenotes{0.5cm}

% VERTICAL SPACE BEFORE RULES
\newlength{\before} 			% A length which will contains the space before the rule
\setlength{\before}{3mm} 		% The space we want to have
\addtolength{\before}{3pt} 		% A compensation for the space decreased by \footnoterule
\Xbeforenotes{\before} 			% And so, we configure reledmac.

% VERTICAL SPACE AFTER RULES
\newlength{\after}				% A length which will contains the space before the rule
\setlength{\after}{3mm}			% The space we want to have
\addtolength{\after}{-2.6pt}	% A compensation for the space added by \footnoterule
\Xafterrule{\after}				% And so, we configure reledmac.

\title{reledmac tagging test - 9}

\begin{document}

\beginnumbering
\pstart%
Cum defensionum \edtext{laboribus}{\Cfootnote{first note}} senatoriisque muneribus aut omnino aut magna ex parte essem aliquando liberatus, rettuli me, Brute, te hortante maxime ad ea studia, quae retenta animo, remissa temporibus, longo intervallo intermissa revocavi, et cum omnium artium, quae ad rectam vivendi viam pertinerent, \edtext{ratio}{\Afootnote{second note}} et disciplina studio sapientiae, quae philosophia dicitur, contineretur, hoc mihi Latinis litteris \edtext{inlustrandum}{\Cfootnote{third note}} putavi, non quia \edtext{philosophia}{\Bfootnote{fourth note}} Graecis et litteris et doctoribus percipi non posset, sed meum semper iudicium fuit omnia nostros aut invenisse per se sapientius quam Graecos aut accepta ab illis fecisse meliora, quae quidem digna statuissent, in quibus elaborarent.
\pend

\pstart%
Cum defensionum \edtext{laboribus}{\Cfootnote{first note}} senatoriisque muneribus aut omnino aut magna ex parte essem aliquando liberatus, rettuli me, Brute, te hortante maxime ad ea studia, quae retenta animo, remissa temporibus, longo intervallo intermissa revocavi, et cum omnium artium, quae ad rectam vivendi viam pertinerent, \edtext{ratio}{\Afootnote{second note}} et disciplina studio sapientiae, quae philosophia dicitur, contineretur, hoc mihi Latinis litteris \edtext{inlustrandum}{\Cfootnote{third note}} putavi, non quia \edtext{philosophia}{\Bfootnote{fourth note}} Graecis et litteris et doctoribus percipi non posset, sed meum semper iudicium fuit omnia nostros aut invenisse per se sapientius quam Graecos aut accepta ab illis fecisse meliora, quae quidem digna statuissent, in quibus elaborarent.
\pend

\pstart%
Cum defensionum \edtext{laboribus}{\Cfootnote{first note}} senatoriisque muneribus aut omnino aut magna ex parte essem aliquando liberatus, rettuli me, Brute, te hortante maxime ad ea studia, quae retenta animo, remissa temporibus, longo intervallo intermissa revocavi, et cum omnium artium, quae ad rectam vivendi viam pertinerent, \edtext{ratio}{\Afootnote{second note}} et disciplina studio sapientiae, quae philosophia dicitur, contineretur, hoc mihi Latinis litteris \edtext{inlustrandum}{\Cfootnote{third note}} putavi, non quia \edtext{philosophia}{\Bfootnote{fourth note}} Graecis et litteris et doctoribus percipi non posset, sed meum semper iudicium fuit omnia nostros aut invenisse per se sapientius quam Graecos aut accepta ab illis fecisse meliora, quae quidem digna statuissent, in quibus elaborarent.
\pend

\pstart%
Cum defensionum \edtext{laboribus}{\Cfootnote{first note}} senatoriisque muneribus aut omnino aut magna ex parte essem aliquando liberatus, rettuli me, Brute, te hortante maxime ad ea studia, quae retenta animo, remissa temporibus, longo intervallo intermissa revocavi, et cum omnium artium, quae ad rectam vivendi viam pertinerent, \edtext{ratio}{\Afootnote{second note}} et disciplina studio sapientiae, quae philosophia dicitur, contineretur, hoc mihi Latinis litteris \edtext{inlustrandum}{\Cfootnote{third note}} putavi, non quia \edtext{philosophia}{\Bfootnote{fourth note}} Graecis et litteris et doctoribus percipi non posset, sed meum semper iudicium fuit omnia nostros aut invenisse per se sapientius quam Graecos aut accepta ab illis fecisse meliora, quae quidem digna statuissent, in quibus elaborarent.
\pend

\pstart%
Cum defensionum \edtext{laboribus}{\Cfootnote{first note}} senatoriisque muneribus aut omnino aut magna ex parte essem aliquando liberatus, rettuli me, Brute, te hortante maxime ad ea studia, quae retenta animo, remissa temporibus, longo intervallo intermissa revocavi, et cum omnium artium, quae ad rectam vivendi viam pertinerent, \edtext{ratio}{\Afootnote{second note}} et disciplina studio sapientiae, quae philosophia dicitur, contineretur, hoc mihi Latinis litteris \edtext{inlustrandum}{\Cfootnote{third note}} putavi, non quia \edtext{philosophia}{\Bfootnote{fourth note}} Graecis et litteris et doctoribus percipi non posset, sed meum semper iudicium fuit omnia nostros aut invenisse per se sapientius quam Graecos aut accepta ab illis fecisse meliora, quae quidem digna statuissent, in quibus elaborarent.
\pend

\pstart%
Cum defensionum \edtext{laboribus}{\Cfootnote{first note}} senatoriisque muneribus aut omnino aut magna ex parte essem aliquando liberatus, rettuli me, Brute, te hortante maxime ad ea studia, quae retenta animo, remissa temporibus, longo intervallo intermissa revocavi, et cum omnium artium, quae ad rectam vivendi viam pertinerent, \edtext{ratio}{\Afootnote{second note}} et disciplina studio sapientiae, quae philosophia dicitur, contineretur, hoc mihi Latinis litteris \edtext{inlustrandum}{\Cfootnote{third note}} putavi, non quia \edtext{philosophia}{\Bfootnote{fourth note}} Graecis et litteris et doctoribus percipi non posset, sed meum semper iudicium fuit omnia nostros aut invenisse per se sapientius quam Graecos aut accepta ab illis fecisse meliora, quae quidem digna statuissent, in quibus elaborarent.
\pend

\endnumbering

\end{document}

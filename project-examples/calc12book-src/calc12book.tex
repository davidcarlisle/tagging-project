\DocumentMetadata{
  lang        = en,
  pdfversion  = 2.0,
  pdfstandard = ua-2,
  pdfstandard = a-4,
  tagging-setup={math/setup=mathml-SE}
}


\documentclass[paper=letter,fontsize=11pt,headings=big,chapterprefix=false,
appendixprefix,twoside,titlepage,open=any,numbers=noenddot,
index=totoc]{scrbook}
\usepackage[standardsections]{scrhack}
\usepackage[,marginparwidth=0pt,marginparsep=0pt,headsep=12pt,inner=1.2in,
outer=1.2in,top=1in,bottom=0.85in,includehead,bindingoffset=0pt,includefoot,
vcentering,hcentering,ignoremp,portrait,letterpaper]{geometry}
\usepackage{graphicx}
\usepackage{float}
\usepackage[intoc]{nomencl}
\usepackage{xfrac}
\usepackage{fancybox}
\usepackage{braket}
\usepackage{pifont}
\usepackage{phaistos}
\usepackage{dingbat}
% Disable the \checkmark command from dingbat, to avoid conflict with amssymb's \checkmark
\let\checkmark\relax
\usepackage{cancel}
\ifx\directlua\undefined
  \usepackage{amssymb}
  \usepackage{amsmath}
\else
  \usepackage{unicode-math}
  \def\wideparen#1{\overparen{#1}}
\fi
\usepackage{amsthm}
\usepackage{relsize}
\usepackage{bigints}
\usepackage{mathtools}
%\usepackage[dvips,gray]{xcolor}
\usepackage[gray]{xcolor}
\usepackage{colortbl}
\usepackage[english]{babel}
\usepackage{fancyvrb}
\usepackage[titletoc]{appendix}
\usepackage{imakeidx}
\usepackage[unbalanced]{idxlayout}
\ifx\directlua\undefined
  \usepackage[T1]{fontenc}
  \usepackage{fouriernc}
\else
  \usepackage{fourier-otf}
\fi
\usepackage[scaled=0.80]{beramono}
\usepackage{bm}
\usepackage{ragged2e}
\usepackage{paralist}
% not good for tagging \usepackage{mdwlist}
\makeatletter
\ExpandArgs{c}\def{c@*}{\value\@currentcounter}
\def\resume#1[#2]{\begin{#1}[resume,#2]}
\def\suspend#1{\end{#1}}
\makeatother

\usepackage{multicol}
\usepackage[labelsep=none]{caption}
\usepackage{subfig}
\usepackage{tikz}
\usepackage{gnuplot-lua-tikz}
\usetikzlibrary{arrows}
\usetikzlibrary{patterns}
\usetikzlibrary{decorations}
\usetikzlibrary{intersections}
\usetikzlibrary{matrix}
\usetikzlibrary{shadows}
\usetikzlibrary{calc}
\usetikzlibrary{backgrounds}
\usetikzlibrary{decorations.pathreplacing}
\usetikzlibrary{decorations.markings}
\usetikzlibrary{decorations.pathmorphing}
\usepackage{tkz-euclide}
%\usetkzobj{all}
\usepackage{picins}
\usepackage{framed}
\usepackage{nextpage}
\usepackage{colortbl}
\usepackage{titlesec}
\usepackage{fancyhdr}
\usepackage{listings}
\lstloadlanguages{Java,C,Octave,Gnuplot,Python}
\usepackage{clrscode}
\usepackage[pdfproducer={GPL Ghostscript 9.55.0},pdfcreator={TeX Live 2020},
            pdfauthor={Michael Corral},pdftitle={Elementary Calculus},
            pdfsubject={Calculus},pdfkeywords={calculus},
            pdfdisplaydoctitle=true,bookmarks,pdfborder={0 0 0},
            bookmarksnumbered=true,colorlinks=false]{hyperref}
\usepackage{breakurl}
\definecolor{captioncolor}{HTML}{000000}
\definecolor{linecolor}{HTML}{0074C8}
\definecolor{linecolor2}{HTML}{E14B4B}
\definecolor{linecolor3}{HTML}{657822}
\definecolor{fillcolor}{cmyk}{0.1,0.05,0,0}
\definecolor{fillcolor2}{HTML}{96CBE9}
\definecolor{brickcolor}{HTML}{F0D8B2}
\definecolor{blockcolor}{HTML}{B6B6B6}
\definecolor{groundcolor}{HTML}{E4D8C5}
\definecolor{earthcolor}{HTML}{C5FFFF}
\definecolor{watercolor}{cmyk}{0.1,0.05,0,0}
\definecolor{codecolor}{HTML}{FFF7E0}
\definecolor{linenumcolor}{HTML}{8C8C8C}
\definecolor{headercolor}{HTML}{DEDEDE}
\definecolor{insideo}{HTML}{798084}
\definecolor{insidei}{HTML}{F0F0F0}
\definecolor{outer}{HTML}{424296}
\definecolor{inner}{HTML}{D8D8FF}
\definecolor{planecolor}{HTML}{5AB487}
\definecolor{conicfillcolor}{HTML}{31805A}
\definecolor{spherecolor}{HTML}{80DCFF}
\definecolor{titleblue}{HTML}{0871B0}
\DeclareCaptionFont{blue}{\color{captioncolor}}
\definecolor{mred}{HTML}{EC1E0B}
\captionsetup{labelfont={small,bf,blue},font={small,blue},textfont={small,blue}}
\providecommand{\lineacross}{\hrule width \textwidth height 0.5pt}

% Set the sans serif font to Helvetica (for chapter and section headings)
\ifx\directlua\undefined
  \renewcommand\sfdefault{phv}
\else
  \setsansfont{TeX Gyre Heros}
\fi

\newcommand{\startexercises}{
 \centerline{\setlength{\fboxsep}{3pt}\fbox{\textsf{\textbf{\large Exercises}}}}
}
\newcommand{\probs}[1]{\par\noindent\textsf{\textbf{\large #1}}}
\newcommand{\hrulethick}{\leaders\hrule height 4pt depth -2pt\hfill}
\newcommand{\divider}{\noindent{\color{black}\rule{\linewidth}{2pt}}}
\setlength{\multicolsep}{2.5mm}
\setlength{\leftmargini}{1.4em}
\theoremstyle{definition}
\newtheoremstyle{itexmp}%
 {\topsep}
 {\topsep}
 {\normalfont\small}
 {0pt}
 {\itshape\bfseries}{}
 { }
 {#1 \textbf{#2} \enskip {\color{black}\hrulethick}\\[6pt]}
\theoremstyle{itexmp}
\newtheorem{exmp}{Example}[chapter]
\newcommand{\statedefn}[2]{
 \definecolor{defncolor}{HTML}{E6F0FF}
 \definecolor{defnrulecolor}{HTML}{0074C8}
 \begin{center}
 \tikzstyle{mybox} = [draw=defnrulecolor,fill=defncolor,line width=1.2pt,
  rectangle,inner sep=8pt]
 \begin{tikzpicture}
  \node [mybox] (box){%
   \begin{minipage}[t]{0.96\textwidth}\label{#1}{#2}\end{minipage}
  };
 \end{tikzpicture}
 \end{center}
}
\newcommand{\statethm}[2]{
 \definecolor{thmcolor}{HTML}{FFFFD8}
 \definecolor{thmrulecolor}{HTML}{9B5900}
 \begin{center}
 \tikzstyle{mybox} = [draw=thmrulecolor,fill=thmcolor,line width=1.2pt,
  rectangle,inner sep=8pt]
 \begin{tikzpicture}
  \node [mybox] (box){%
   \begin{minipage}[t]{0.96\textwidth}\label{#1}{#2}\end{minipage}
  };
 \end{tikzpicture}
 \end{center}
}
\newcommand{\statecor}[2]{
 \definecolor{corcolor}{HTML}{FAF2FF}
 \definecolor{corrulecolor}{HTML}{4B3880}
 \begin{center}
 \tikzstyle{mybox} = [draw=corrulecolor,fill=corcolor,line width=1.2pt,
  rectangle,inner sep=8pt]
 \begin{tikzpicture}
  \node [mybox] (box){%
   \begin{minipage}[t]{0.96\textwidth}\label{#1}{#2}\end{minipage}
  };
 \end{tikzpicture}
 \end{center}
}
\newcommand{\statecomment}[2][0.96\textwidth]{
 \begin{center}
 \tikzstyle{mybox} = [draw=green!70!black,fill=green!12,line width=1.2pt,
  rectangle,inner sep=8pt]
 \begin{tikzpicture}
  \node [mybox] (box){%
   \begin{minipage}{#1}{#2}\end{minipage}
  };
 \end{tikzpicture}
 \end{center}
}

\newcommand*\widefbox[1]{\fbox{\hspace{1em}#1\hspace{1em}}}
\renewcommand{\CancelColor}{\color{red}}
\newcommand\myclearpage{\cleartooddpage
 [\thispagestyle{empty}]}
\providecommand{\abs}[1]{\lvert\mspace{1mu}#1\mspace{1mu}\rvert}
\providecommand{\Abs}[1]{\bigl\lvert\mspace{1mu}#1\mspace{1mu}\bigr\rvert}
\providecommand{\ABS}[1]{\Biggl\lvert\mspace{1mu}#1\mspace{1mu}\Biggr\rvert}
\providecommand{\norm}[1]{\lVert\mspace{1mu}#1\mspace{1mu}\rVert}
\providecommand{\Norm}[1]{\bigl\lVert\mspace{1mu}#1\mspace{1mu}\bigr\rVert}
\providecommand{\NORM}[1]{\Biggl\lVert\mspace{1mu}#1\mspace{1mu}\Biggr\rVert}
\providecommand{\avg}[1]{\langle\mspace{1mu}#1\mspace{1mu}\rangle}
\providecommand{\Avg}[1]{\left\langle\mspace{1mu}#1\mspace{1mu}\right\rangle}
\providecommand{\ssub}[2]{#1_{\scriptscriptstyle #2}}
\providecommand{\ssubsum}[3]{#1_{\scriptscriptstyle #3} + 
 #2_{\scriptscriptstyle #3}}
\providecommand{\dy}{d\!y}
\providecommand{\df}{d\negmedspace f}
\providecommand{\dg}{d\negmedspace g}
\providecommand{\dP}{d\negmedspace P}
\providecommand{\dV}{d\negmedspace V}
\providecommand{\dT}{d\negmedspace T}
\providecommand{\dG}{d\negmedspace G}
\providecommand{\dA}{d\negmedspace A}
\providecommand{\dr}{d\!r}
\providecommand{\du}{d\!u}
\providecommand{\dv}{d\!v}
\providecommand{\dx}{d\!x}
\providecommand{\dt}{d\!t}
\providecommand{\dm}{d\!m}
\providecommand{\dtheta}{d\!\theta}
\providecommand{\da}{d\!a}
\providecommand{\ds}{d\!s}
\providecommand{\dalpha}{d\!\alpha}
\providecommand{\dphi}{d\!\phi}
\providecommand{\dydx}{\frac{d\negmedspace y}{d\!x}}
\providecommand{\Dydx}{\dfrac{d\negmedspace y}{d\!x}}
\providecommand{\dxdy}{\frac{d\negmedspace x}{d\!y}}
\providecommand{\dfdx}{\frac{d\negmedspace f}{d\!x}}
\providecommand{\dgdx}{\frac{d\negmedspace g}{d\!x}}
\providecommand{\dydu}{\frac{d\negmedspace y}{d\!u}}
\providecommand{\dfdu}{\frac{d\negmedspace f}{d\!u}}
\providecommand{\dudx}{\frac{d\negmedspace u}{d\!x}}
\providecommand{\dvdu}{\frac{d\negmedspace v}{d\!u}}
\providecommand{\dfdv}{\frac{d\negmedspace f}{d\!v}}
\providecommand{\dfdt}{\frac{d\negmedspace f}{d\!t}}
\providecommand{\dxdt}{\frac{d\negmedspace x}{d\!t}}
\providecommand{\dydt}{\frac{d\negmedspace y}{d\!t}}
\providecommand{\Dxdt}{\dfrac{d\negmedspace x}{d\!t}}
\providecommand{\Dydt}{\dfrac{d\negmedspace y}{d\!t}}
\providecommand{\dsdt}{\frac{d\negmedspace s}{d\!t}}
\providecommand{\drdt}{\frac{d\negmedspace r}{d\!t}}
\providecommand{\dvdt}{\frac{d\negmedspace v}{d\!t}}
\providecommand{\dVdt}{\frac{d\negmedspace V}{d\!t}}
\providecommand{\ddx}{\frac{d}{d\!x}}
\providecommand{\Ddx}{\dfrac{d}{d\!x}}
\providecommand{\ddt}{\frac{d}{d\!t}}
\providecommand{\Ddt}{\dfrac{d}{d\!t}}
\providecommand{\ddu}{\frac{d}{d\!u}}
\providecommand{\ddy}{\frac{d}{d\!y}}
\providecommand{\Reals}{\mathbb{R}}
\providecommand{\Complex}{\mathbb{C}}
\providecommand{\Rationals}{\mathbb{Q}}
\providecommand{\Naturals}{\mathbb{N}}
\providecommand{\Integers}{\mathbb{Z}}
\providecommand{\Degrees}[0]{\ensuremath{^\circ}}
\providecommand{\ival}[2]{\lbrack #1,#2 \rbrack}
\providecommand{\lival}[2]{\lbrack #1,#2 )}
\providecommand{\rival}[2]{( #1,#2 \rbrack}
\providecommand{\seq}[1]{\left\lbrace\mspace{3mu}#1\mspace{3mu}\right\rbrace}
\providecommand{\bigsum}[2]{\displaystyle\mathlarger{\sum}\limits_{#1}^{#2}}
\DeclareMathOperator{\sech}{sech}
\DeclareMathOperator{\csch}{csch}
\DeclareMathOperator{\arccot}{arccot}
\DeclareMathOperator{\arccsc}{arccsc}
\DeclareMathOperator{\arcsec}{arcsec}
\DeclareMathOperator{\sgn}{sgn}
\DeclareMathOperator{\sn}{sn}
\DeclareMathOperator{\cn}{cn}
\DeclareMathOperator{\dn}{dn}
\newcommand*{\sector}{%
  \begin{pgfpicture}
   \pgfpathmoveto{\pgforigin}%
   \pgfpathlineto{\pgfpointpolar{45}{1ex}}%
   \pgfarc{45}{-45}{1ex}%
   \pgfpathclose%
   \pgfsetlinewidth{0.4pt}%
   \pgfusepath{stroke}%
  \end{pgfpicture}%
}
\newcommand*{\hypsector}{%
  \begin{pgfpicture}
   \pgfpathmoveto{\pgfpoint{2ex}{0ex}}%
   \pgfarc{225}{135}{1ex}%
   \pgfpathlineto{\pgfpoint{0ex}{0.7ex}}%
   \pgfpathclose%
   \pgfsetlinewidth{0.4pt}%
   \pgfusepath{stroke}%
  \end{pgfpicture}%
}
\newenvironment{proofbar}{%
  \def\FrameCommand{{\color{black}\vrule width 1pt \hspace{5pt}}}%
  \MakeFramed {\advance\hsize-\width \FrameRestore}}%
 {\endMakeFramed}
\allowdisplaybreaks[1]
\numberwithin{figure}{section}
\setlength\topmargin{0pt}
\setlength\headsep{10pt}
\setlength\textheight{202mm}
%Change indent after chapter number in TOC from 1.5em to 2.2em, and from 2.3 to
%3.0 for sections (see Ch.10)
\makeatletter
\renewcommand*\l@chapter[2]{%
  \ifnum \c@tocdepth >\m@ne
    \addpenalty{-\@highpenalty}%
    \vskip 1.0em \@plus\p@
    \setlength\@tempdima{2.2em}%
    \if@tocleft
      \ifx\toc@l@number\@empty\else
        \setlength\@tempdima{0\toc@l@number}%
      \fi
    \fi
    \begingroup
      \parindent \z@ \rightskip \@pnumwidth
      \parfillskip -\@pnumwidth
      \leavevmode \sectfont
      \advance\leftskip\@tempdima
      \hskip -\leftskip
      #1\nobreak\hfil \nobreak\hb@xt@\@pnumwidth{\hss #2}\par
      \penalty\@highpenalty
    \endgroup
  \fi}
\renewcommand*\l@section{\@dottedtocline{1}{2.2em}{3.0em}}
\makeatother
\makeindex[intoc,options=-s myindex]
\setlength{\headheight}{22.29259pt}\addtolength{\topmargin}{-0.02927pt}
\begin{document}
\VerbatimFootnotes
\deffootnote[1em]{0em}{1em}{\textsuperscript{\thefootnotemark}}
\frontmatter
\pagestyle{empty}
%Put the front cover here
% \thisfancypage{\setlength{\fboxsep}{1pt}\setlength\fboxrule{1.5pt}\fbox}{}

% \includegraphics[scale=1.0]{calc12bookcover.eps}\vspace{7mm}

% \centerline{\includegraphics[scale=0.86]{coverplot}}
\newgeometry{left=0cm,top=0cm,bottom=0cm,right=0.5in}
\begin{flushright}
\begin{tikzpicture}[remember picture,overlay]
 \node[right] at (-15,-6) {\includegraphics[scale=1.0]{calcbookcovertitle}};
 \node[right] at (-16.5,-15) {\includegraphics[scale=0.86]{coverplot}};
 \node[right] at (-4.5,-23) {\includegraphics[scale=1.0]{calcbookcoverbottom}};
 \node[shape=rectangle,fill=mred,minimum height=\paperheight,minimum width=1.8in,anchor=west] at (current page.west) {};
\end{tikzpicture}
\end{flushright}
\restoregeometry
\setlength{\headheight}{22.29259pt}\addtolength{\topmargin}{-0.02927pt}
% Put the title page here
\title{Elementary Calculus}
\author{\textsf{\textbf{Michael Corral}}}
\date{\large \textsf{\textsl{Schoolcraft College}}}
%Put the author/copyright info here
\uppertitleback{\emph{About the author}:\\
Michael Corral is an Adjunct Faculty member of the Department of Mathematics at
Schoolcraft College. He received a B.A. in Mathematics from the University of
California, Berkeley, and received an M.A. in Mathematics and an M.S. in
Industrial \& Operations Engineering from the University of Michigan.\\\\
This text was typeset in \LaTeX\medspace with the \textsf{KOMA-Script} bundle,
using the GNU Emacs text editor on a
Fedora Linux system. The graphics were created using TikZ and Gnuplot.}
\lowertitleback{Copyright \copyright ~2020 ~Michael Corral.\\
Permission is granted to copy, distribute and/or modify this document
under the terms of the GNU Free Documentation License, Version 1.3
or any later version published by the Free Software Foundation;
with no Invariant Sections, no Front-Cover Texts, and no Back-Cover
Texts.}
\maketitle
\pagestyle{fancy}
\addtokomafont{footnotelabel}{\color{linecolor}}
\addtokomafont{footnotereference}{\color{linecolor}}
\renewcommand{\chaptermark}[1]{\markboth{#1}{}}
\renewcommand{\sectionmark}[1]{\markright{#1}}
\renewcommand{\qedsymbol}{\textsf{\textbf{\textsc{\small{QED}}}}}
\fancyhf{}
\setlength{\headheight}{22.6pt}\addtolength{\topmargin}{-0.3pt}
\renewcommand{\headrulewidth}{0pt}
\newlength{\fminilength}%
\setlength{\fminilength}{\textwidth-2\fboxsep-2\fboxrule}%
\fancyhf{}
\fancyhead[LE]{\small\thepage}
\fancyhead[CE]{\small\scshape\leftmark}
\fancyhead[RO]{\small\thepage}
\fancyhead[CO]{\small\scshape\leftmark}
%Put the preface here
%\titleformat{\chapter}[display]
%  {\filcenter\huge\selectfont\bfseries\sffamily}
%  {}{0mm}{}[\vspace{3mm}\titlerule]
%\renewcommand*{\chapterformat}{\filcenter\huge\selectfont\bfseries\sffamily}
\titleformat{\chapter}[display]
  {\filcenter\huge\selectfont\bfseries\sffamily}
  {}{0mm}{\titlerule[1.5pt]\vspace{3mm}}[\vspace{3mm}{\titlerule[1.5pt]}]
\titleformat*{\section}{\Large\bfseries\sffamily}
\include{calc12book-preface}
%\thispagestyle{empty}
\pdfbookmark[0]{\contentsname}{toc}
%Put the table of contents here
%\titlecontents{chapter}
%[0pt]
%{\addvspace{1.5pc}%
%\filright}
%{\selectfont\bfseries\sffamily\MakeUppercase{\chaptername} \thecontentslabel\\*[.2pc]%
%\large\selectfont\bfseries\sffamily}
%{}
%{\hfill\contentspage} % That is, without page number
%[\addvspace{.5pc}]
\tableofcontents
%Put a completely blank page here since the TOC has an odd number of pages
\clearpage{\pagestyle{empty}\include{calc12book-greek}}

\mainmatter
\fancyhf{}
\fancyhead[LE]{\tikzstyle{headbox} = [fill=headercolor,line width=0pt,rectangle,
 rounded corners]
 \begin{tikzpicture}
  \node [headbox] (box){%
   \begin{minipage}{\fminilength}%
     \sffamily{\bfseries\thepage\qquad Chapter \thechapter}\enskip $\bullet$
      \enskip\leftmark\hfill{\bfseries\S\thesection}
    \end{minipage}%
  };
 \end{tikzpicture}}
\fancyhead[LO]{\tikzstyle{headbox} = [fill=headercolor,line width=0pt,rectangle,
 rounded corners]
 \begin{tikzpicture}
  \node [headbox] (box){%
   \begin{minipage}{\fminilength}%
     \sffamily {}\hfill\rightmark\enskip $\bullet$ \enskip{\bfseries{Section
     \thesection\qquad\thepage}}
    \end{minipage}%
  };
 \end{tikzpicture}}
%\titleformat{\chapter}[display]
%  {\filcenter\LARGE\selectfont\bfseries\sffamily}
%  {CHAPTER \thechapter}{3mm}{\titlerule\vspace{3mm}\huge}
\titleformat{\chapter}[display]
  {\filcenter\LARGE\selectfont\bfseries\sffamily}
  {\titlerule[1.5pt]\vspace{3mm}\MakeUppercase{\chaptertitlename}
   \thechapter}{3mm}
  {\titlerule[1.5pt]\vspace{3mm}\huge}
%Put the main chapters here
\include{calc12book-chapter1}
\include{calc12book-chapter2}
\include{calc12book-chapter3}
\include{calc12book-chapter4}
\include{calc12book-chapter5}
\include{calc12book-chapter6}
\include{calc12book-chapter7}
\include{calc12book-chapter8}
\include{calc12book-chapter9}
\newpage
%Put the bibliography here
\fancyhf{}
\fancyhf{}
\fancyhead[LE]{\tikzstyle{headbox} = [fill=headercolor,line width=0pt,rectangle,
 rounded corners]
 \begin{tikzpicture}
  \node [headbox] (box){%
   \begin{minipage}{\fminilength}%
     \sffamily{\bfseries\thepage\qquad \enskip\leftmark}
    \end{minipage}%
  };
 \end{tikzpicture}}
\fancyhead[LO]{\tikzstyle{headbox} = [fill=headercolor,line width=0pt,rectangle,
 rounded corners]
 \begin{tikzpicture}
  \node [headbox] (box){%
   \begin{minipage}{\fminilength}%
     \sffamily {}\hfill{\bfseries{\leftmark\qquad\thepage}}
    \end{minipage}%
  };
 \end{tikzpicture}}
%\titleformat{\chapter}[display]
%  {\filcenter\huge\selectfont\bfseries\sffamily}
%  {}{0mm}{\titlerule[1.5pt]\vspace{3mm}}[\vspace{3mm}{\titlerule[1.5pt]}]
%\include{calc12book-biblio}
%Put appendices here
\titleformat{\chapter}[display]
  {\filcenter\LARGE\selectfont\bfseries\sffamily}
  {\titlerule[1.5pt]\vspace{3mm}\MakeUppercase{\chaptertitlename}
   \thechapter}{3mm}
  {\titlerule[1.5pt]\vspace{3mm}\huge}
\titleformat*{\section}{\Large\bfseries\sffamily}
\fancyhf{}
\fancyhead[LE]{\tikzstyle{headbox} = [fill=headercolor,line width=0pt,rectangle,rounded corners]
 \begin{tikzpicture}
  \node [headbox] (box){%
   \begin{minipage}{\fminilength}%
     \sffamily{\bfseries\thepage\qquad \enskip Appendix A:\quad\leftmark}
    \end{minipage}%
  };
 \end{tikzpicture}}
\fancyhead[LO]{\tikzstyle{headbox} = [fill=headercolor,line width=0pt,rectangle,rounded corners]
 \begin{tikzpicture}
  \node [headbox] (box){%
   \begin{minipage}{\fminilength}%
     \sffamily {}\hfill{\bfseries{Appendix A:\quad\leftmark\qquad\thepage}}
    \end{minipage}%
  };
 \end{tikzpicture}}
\begin{appendices}
 \include{calc12book-appendixa}
\end{appendices}
\newpage
%Put the GNU FDL here
\titleformat{\chapter}[display]
  {\filcenter\huge\selectfont\bfseries\sffamily}
  {}{0mm}{\titlerule[1.5pt]\vspace{3mm}}[\vspace{3mm}{\titlerule[1.5pt]}]
\fancyhead[LE]{\tikzstyle{headbox} = [fill=headercolor,line width=0pt,rectangle,rounded corners]
 \begin{tikzpicture}
  \node [headbox] (box){%
   \begin{minipage}{\fminilength}%
     \sffamily{\bfseries\thepage\qquad \enskip\leftmark}
    \end{minipage}%
  };
 \end{tikzpicture}}
\fancyhead[LO]{\tikzstyle{headbox} = [fill=headercolor,line width=0pt,rectangle,rounded corners]
 \begin{tikzpicture}
  \node [headbox] (box){%
   \begin{minipage}{\fminilength}%
     \sffamily {}\hfill{\bfseries{\leftmark\qquad\thepage}}
    \end{minipage}%
  };
 \end{tikzpicture}}
\include{fdl-1.3}
%Put the history section here
\addchap{History}
This section contains the revision history of the book. For persons making
modifications to the book, please record the pertinent information here,
following the format in the first item below.

\begin{enumerate}

\item VERSION: 0.1
\begin{description}
\item[Date:] 2016-01-24
\item[Author(s):] Michael Corral
\item[Title:] Elementary Calculus
\item[Modification(s):] Initial version
\end{description}

\item VERSION: 0.5
\begin{description}
\item[Date:] 2020-05-27
\item[Author(s):] Michael Corral
\item[Title:] Elementary Calculus
\item[Modification(s):] Numerous corrections of typos and errors, as well as
enhancements of some sections and more exercises.
\end{description}

\item VERSION: 0.5a
\begin{description}
\item[Date:] 2020-06-03
\item[Author(s):] Michael Corral
\item[Title:] Elementary Calculus
\item[Modification(s):] Replaced the old Section 5.5 (Average Value of a
Function) with the new Section 5.5 (Improper Integrals). Average Value of a
Function will be moved to Chapter 8 in the second half of the book.
\end{description}

\item VERSION: 1.0
\begin{description}
\item[Date:] 2020-12-31
\item[Author(s):] Michael Corral
\item[Title:] Elementary Calculus
\item[Modification(s):] Full version.
\end{description}

\item VERSION: 1.0-tagged
\begin{description}
\item[Date:] 2025-01-19
\item[Author(s):] \LaTeX\ Project
\item[Title:] Elementary Calculus
\item[Modification(s):] Modifications to the markup to produce
  accessible tagged PDF. No changes to textual content.
\end{description}

\end{enumerate}

\backmatter
\clearpage
\phantomsection
%Put the index here
\printindex

%Add a blank page at the end to make the number of pages divisible by 4
\myclearpage
\end{document}

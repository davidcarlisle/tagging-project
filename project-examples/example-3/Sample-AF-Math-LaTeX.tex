
\DocumentMetadata{
 lang=en,
 pdfversion=2.0,
 pdfstandard=ua-2,
 pdfstandard=a-4f,
 uncompress,
 tagging-setup={math/mathml/luamml/load=false}
}
%\tagpdfsetup{math/mathml/write-dummy}
\documentclass{article}

\setcounter{secnumdepth}{-1}

\usepackage[sfdefault]{noto-sans}
\usepackage{notomath}
\input{glyphtounicode-ntx}

\AtBeginDocument{\prehyphenchar"AD }

\begin{document}


\title{Sample MathML Expressions}
\maketitle

\section{Color matrices}

For each color space is defined a $3\times 3$ forward matrix $MX_i$
for converting an RGB triplet to a CIE XYZ tristimulus value.  This
conversion takes the form $XYZ = MX_i \cdot RGB$.  The forward
matrices $MX$ are calculated from the color primaries and white point
using the equation:
\[
MX_i =
\left[
  \begin{Bmatrix} r_i \\ g_i \\ b_i \end{Bmatrix}
  \times
  \left(
  \frac{w_i}{w_i \cdot y}
  \cdot
  \begin{Bmatrix} r_i \\ g_i \\ b_i \end{Bmatrix}^{-1}
  \right)
\right]^T
\]
where
\begin{description}
\item[\mdseries $i$]
  is one of SD, HD and UHD (HDR uses same matrix as UHD),
  
\item[\mdseries $r_i, g_i, b_i$ and $w_i$]

  are the $xyz$ triplets for the red, green, blue primaries and the
  white point of the color space $i$,
  
\item[\mdseries $w_i \cdot y$] is the $y$ value
  of $w_i$, an $xyz$ value being an $xy$ chromaticity coordinate
  appended with $z = 1 - x - y$, and

\item[\mdseries $\cdot$ (dot)]
  indicates the matrix product or inner product.
\end{description}

The resulting matrices are
\begin{align*}
  MX_{HD} &=
  \begin{Bmatrix}
    0.412391 & 0.357584 & 0.180481 \\
    0.212639 & 0.715169 & 0.072192 \\
    0.019331 & 0.119195 & 0.950532 
  \end{Bmatrix}
\\
  MX_{SD} &=
  \begin{Bmatrix}
    0.430554 & 0.341550 & 0.178352 \\
    0.222004 & 0.706655 & 0.071341 \\
    0.020182 & 0.129553 & 0.939322 
  \end{Bmatrix}
\\
  MX_{UHD} &=
  \begin{Bmatrix}
    0.636958 & 0.144617 & 0.168881 \\
    0.2627 & 0.677998   & 0.059302 \\
    0      & 0.028073   & 1.060990    
  \end{Bmatrix}
\end{align*}

The matrices for converting from linear RGB values in HD space to linear RGB values in the SD or
UHD color space are obtained by combining forward and inverse matrices. The numerical values for
the matrices are:
\begin{align*}
  M_S &= (MX_{SD})^{-1} \cdot MX_{HD} =
  \begin{Bmatrix}
     0.957815 & 0.0421852 & 0 \\
     0 & 1. & 0 \\
     0 & -0.0119341 & 1.01193
  \end{Bmatrix}
\\
  M_U &= (MX_{UHD})^{-1} \cdot MX_{HD} =
  \begin{Bmatrix}
    0.627404 & 0.329283 & 0.043313 \\
    0.069098 & 0.919540 & 0.011362 \\
    0.016391 & 0.088013 & 0.895595
  \end{Bmatrix}
\end{align*}

\section{Transfer functions}

The domain and range are $[0,1]$ for all transfer functions used in this paper.

\end{document}



\documentclass[a4paper]{article}

\usepackage{unicode-math}
\usepackage{parskip}

\usepackage{unicode-math}

\title{Aligned equations}
\author{\LaTeX\ Team}
\date{February 2025}

\addtolength\textheight{6cm}
\addtolength\topmargin{-4cm}
\begin{document}

\maketitle



Simple \texttt{align}
\begin{align}
  a+b&=x^2+x+2\\
    b&=x
\end{align}



Simple \texttt{align*}
\begin{align*}
  c+d&=x^2+x+2\\
    d&=x
\end{align*}


\texttt{aligned} in \texttt{equation}
\begin{equation}
\begin{aligned}
  e+f&=x^2+x+2\\
    f&=x
\end{aligned}
\end{equation}

\texttt{split} in \texttt{align}
\begin{align}
\begin{split}
  x&= a  +  b\\
   &\quad + c
\end{split}\\
  y&=d
\end{align}

\texttt{split} in \texttt{equation}
\begin{equation}
\begin{split}
 a & = b+b\\
   &\quad + c
\end{split} 
\end{equation}

\texttt{split} in \texttt{displaymath}
\[
\begin{split}
 a & = b-b\\
   &\quad + c
\end{split} 
\]

\texttt{split} in \texttt{gather} 
\begin{gather}
\begin{split}
 a & = b\times b\\
   &\quad + c
\end{split}\\
x=1
\end{gather}


\texttt{notag} in  \texttt{align}
\begin{align}
  a+b&=y^2+y+2\notag\\
    b&=y
\end{align}


\texttt{tag} in  \texttt{align*}
\begin{align*}
  r+s&=y^2+y+2\\
    t&=y\tag{\dagger}
\end{align*}


\clearpage

Two groups in \texttt{align}
\begin{align}
  aaa&=bbb &ccc&=ddd\\
    a&=b   &  c&=d
\end{align}

Two groups in \texttt{alignat}
\begin{alignat}{2}
  aaa&=bbb &ccc&=ddd\\
    a&=b   &  c&=d
\end{alignat}

Two groups in \texttt{flalign}
\begin{flalign}
  aaa&=bbb &ccc&=ddd\\
    a&=b   &  c&=d
\end{flalign}


\end{document}


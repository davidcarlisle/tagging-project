
\DocumentMetadata{uncompress,testphase={latest},pdfversion=2.0,
 }

\ExplSyntaxOn
\group_begin:
\pdfdict_put:nnn {l_pdffile/Filespec} {AFRelationship}{/Supplement}
\pdffile_embed_stream:nnn{
<style>
mtable[columnalign="right ~left"] > mtr > mtd:nth-child(odd) {
padding-right:0;text-align:right;text-align:-moz-right;text-align:-webkit-right;
}
mtable[columnalign="right ~left"] > mtr > mtd:nth-child(even) {
padding-left:0;text-align:left;text-align:-moz-left;text-align:-webkit-left;
}
mtd[intent=":noequationlabel"], mtd[intent=":equationlabel"] {
margin-right: 1em;display:block; position:absolute; right:0em
}
</style>
}{latex-math-css.html}{tag/AFCSStest}
\group_end:
\ExplSyntaxOff

\tagpdfsetup{root-AF=tag/AFCSStest, math/mathml/structelem }

\documentclass[a4paper]{article}

\usepackage{unicode-math}
\usepackage{parskip}

\usepackage{unicode-math}

\title{Aligned equations}
\author{\LaTeX\ Team}
\date{February 2025}

\addtolength\textheight{6cm}
\addtolength\topmargin{-4cm}
\begin{document}

\maketitle



Simple \texttt{align}
\begin{align}
  a+b&=x^2+x+2\\
    b&=x
\end{align}



Simple \texttt{align*}
\begin{align*}
  c+d&=x^2+x+2\\
    d&=x
\end{align*}


\texttt{aligned} in \texttt{equation}
\begin{equation}
\begin{aligned}
  e+f&=x^2+x+2\\
    f&=x
\end{aligned}
\end{equation}

\texttt{split} in \texttt{align}
\begin{align}
\begin{split}
  x&= a  +  b\\
   &\quad + c
\end{split}\\
  y&=d\\
\end{align}

\texttt{split} in \texttt{equation}
\begin{equation}
\begin{split}
 a & = b+b\\
   &\quad + c
\end{split} 
\end{equation}

\texttt{split} in \texttt{[}
\[
\begin{split}
 a & = b-b\\
   &\quad + c
\end{split} 
  \]

\texttt{split} in \texttt{gather} 
\begin{gather}
\begin{split}
 a & = b\times b\\
   &\quad + c
\end{split}\\
x=1
\end{gather}


\end{document}

